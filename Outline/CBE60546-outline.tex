% Created 2021-09-05 Sun 13:18
% Intended LaTeX compiler: pdflatex
\documentclass[11pt]{article}
\usepackage[utf8]{inputenc}
\usepackage{lmodern}
\usepackage[T1]{fontenc}
\usepackage[top=1in, bottom=1.in, left=1in, right=1in]{geometry}
\usepackage{graphicx}
\usepackage{longtable}
\usepackage{float}
\usepackage{wrapfig}
\usepackage{rotating}
\usepackage[normalem]{ulem}
\usepackage{amsmath}
\usepackage{textcomp}
\usepackage{marvosym}
\usepackage{wasysym}
\usepackage{amssymb}
\usepackage{amsmath}
\usepackage[theorems, skins]{tcolorbox}
\usepackage[version=3]{mhchem}
\usepackage[numbers,super,sort&compress]{natbib}
\usepackage{natmove}
\usepackage{url}
\usepackage[cache=false]{minted}
\usepackage[strings]{underscore}
\usepackage[linktocpage,pdfstartview=FitH,colorlinks,
linkcolor=blue,anchorcolor=blue,
citecolor=blue,filecolor=blue,menucolor=blue,urlcolor=blue]{hyperref}
\usepackage{attachfile}
\usepackage{setspace}
\usepackage{geometry}
\geometry{margin=1.0in}
\usepackage{outline}
\usepackage{amsmath}
\usepackage{graphicx}
\usepackage{epstopdf}
\usepackage{fancyhdr}
\usepackage{hyperref}
\usepackage[labelfont=bf]{caption}
\setlength{\headheight}{15.2pt}
\def\dbar{{\mathchar'26\mkern-12mu d}}
\pagestyle{fancy}
\fancyhf{}
\renewcommand{\headrulewidth}{0.5pt}
\renewcommand{\footrulewidth}{0.5pt}
\lfoot{\today}
\cfoot{\copyright\ 2017 W.\ F.\ Schneider}
\rfoot{\thepage}
\lhead{\em{Advanced Chemical Reaction Engineering}}
\rhead{ND CBE 60546}
\setcounter{secnumdepth}{3}
\author{William F. Schneider}
\date{\today}
\title{CBE 60546 Outline}
\begin{document}

\begin{OPTIONS}
\end{OPTIONS}
\section{Lecture 0: Intro to Reaction Engineering}
\label{sec:org336f145}
\begin{enumerate}
\item Reaction engineering
\end{enumerate}
\begin{quote}
Understanding, modeling, designing, using, controlling, analyzing, improving anything in which chemical reactions happen.
\end{quote}
\begin{enumerate}
\item Reaction engineering applications
\begin{enumerate}
\item Traditional
\begin{enumerate}
\item Industrial chemical/petroleum processes
\item Fine chemical/pharmaceutical processes
\item Emerging, eg biorefinergy, shale gas, \url{http://cistar.us}
\end{enumerate}
\item Energy storage, batteries, fuel cells
\item Environmental systems
\begin{enumerate}
\item Atmosphere, lake, bioreactor (water purification), catalytic convertor
\end{enumerate}
\item Biological systems
\begin{enumerate}
\item Cell, organ, body
\end{enumerate}
\item Laboratory reactors - interrogate, quantify
\item Research - improved materials (catalysts), improved processes, understand limitations
\begin{enumerate}
\item Sabatier plot, \url{https://doi.org/10.1038/nchem.121}
\end{enumerate}
\end{enumerate}
\item Course structure
\begin{enumerate}
\item Quantifying chemical reactions
\begin{enumerate}
\item Stoichiometry
\item Thermodynamics - heat flow, direction, equilibrium
\item Kinetics - rates, mechanisms
\end{enumerate}
\item Physical/chemical interactions
\begin{enumerate}
\item Transport, mixing, diffusion resistance, \ldots{}
\end{enumerate}
\item Chemical reactors
\begin{enumerate}
\item Ideal 0 and 1-dimensional
\item Non-ideal
\item Non-isothermal
\item Non-steady state
\item Multiphase
\end{enumerate}
\item Chemical processes (beyond us)
\item Markets (beyond us)
\end{enumerate}
\end{enumerate}

\section{Stoichiometry and reactions}
\label{sec:orgafd6ae8}
\begin{enumerate}
\item Substances
\item Amounts
\begin{enumerate}
\item mass, moles, volumes
\item flow rates
\end{enumerate}
\item compositions
\begin{enumerate}
\item amount/total amount
\end{enumerate}
\item Reactions and stoichiometric coefficients
\begin{enumerate}
\item Advancements \(n_j = \sum_i \nu_{ij} \xi_i\)
\item Limiting reagents
\end{enumerate}
\end{enumerate}
\section{Chemical thermodynamics and equilibria}
\label{sec:orgbf01243}
\begin{enumerate}
\item Chemical reactions \(\sum_j \nu_j A_j = 0\)
\item Thermodynamic potential differences
\begin{enumerate}
\item Standard states
\item Formation reactions
\item Reaction enthalpy \(\Delta H^\circ (T) = \sum_j \vu_j H^\circ_j (T) = \sum_j \nu_j H^\circ_{f,j}(T)\)
\item Reaction entropy \(\Delta S^\circ (T) =  \sum_j \vu_j S^\circ_j (T)\)
\end{enumerate}
\item Equilibrium-closed system
\begin{enumerate}
\item Free energy vs reaction advancement, \(G(\xi,T) = \sum_j n_j\mu_j = \sum_j \left (n_{j0} + \nu_j \xi \right ) \left (\mu_j^\circ(T) + RT \ln a(\xi,T) \right )\)
\item Equilibrium \((\partial G / \partial \xi)_{T,P} = 0\)
\item Equilibrium constants and algebraic solutions
\item Multiple reactions
\end{enumerate}
\item Le'Chatlier principle - system at equilibrium responds to oppose any perturbation
\begin{enumerate}
\item Pressure, composition
\item Temperature: Gibbs-Helmholtz and van't Hoff
\end{enumerate}
\item Equilibrium-open system
\begin{enumerate}
\item Reaction phase diagrams, see \url{http://pubs.acs.org/doi/abs/10.1021/jacs.6b02651} for an example
\item Electrochemical reactions
\end{enumerate}
\item The molecular interpretation
\item Non-ideal activities
\item Surface adsorption
\begin{enumerate}
\item Langmuir
\end{enumerate}
\end{enumerate}



\section{Empirical kinetics}
\label{sec:orge8494f0}
\begin{enumerate}
\item rates
\item reactor mass balance
\item rate expressions
\item rate orders
\item apparent orders
\item integrated rate expressions
\item temperature and Arrhenius expression
\item analyzing reactor data
\end{enumerate}

\section{Molecular basis}
\label{sec:orgc49cd49}
\begin{enumerate}
\item reaction pathway, detailed balance
\item bimolecular, collision theory, TST
\item unimolecular reactions
\end{enumerate}

\section{Mechanisms}
\label{sec:orgb1fcbd5}
\begin{enumerate}
\item QSSA
\item Pre-equilibrium
\end{enumerate}

\section{Heterogeneous reactions}
\label{sec:orgc3d38fa}
\begin{enumerate}
\item adsorption, L-H
\item TPD
\item catalysis
\item Sabatier analysis
\end{enumerate}

\section{Liquid-phase reactions}
\label{sec:orgb9fd19e}

\begin{table}
\begin{center}
    \caption{\large{Equilibrium and Rate Constants}}
   \begin{description}
   \item[Equilibrium Constants] $a~\text{A} + b~\text{B} \rightleftharpoons c~\text{C} + d~\text{D} $
     \begin{eqnarray*}
       K_{eq}(T) &=& e^{\Delta S^\circ(T,V)/k_B}e^{-\Delta H^\circ(T,V)/k_BT}
       \\ \\
            K_c(T) &=&
           \left(\frac{1}{c^\circ}\right)^{\nu_c+\nu_d-\nu_a-\nu_b}\frac{(q_c/V)^{\nu_c}(q_d/V)^{\nu_d}}{(q_a/V)^{\nu_a}(q_b/V)^{\nu_b}}e^{-\Delta
            E(0)\beta}\\ \\
            K_p(T) &=&
          \left(\frac{k_BT}{P^\circ}\right)^{\nu_c+\nu_d-\nu_a-\nu_b}\frac{(q_c/V)^{\nu_c}(q_d/V)^{\nu_d}}{(q_a/V)^{\nu_a}(q_b/V)^{\nu_b}}e^{-\Delta
            E(0)\beta}
\end{eqnarray*}
\item[Unimolecular Reaction] $\text[A] \rightleftharpoons [\text{A} ]^\ddagger
  \rightarrow C$
      \begin{displaymath}
        k(T)=\nu^\ddagger \bar K^\ddagger=\frac{k_B T}{h} \frac{\bar{q}_\ddagger(T)/V}{q_A(T)/V}
          e^{-\Delta E^\ddagger(0)\beta}
      \end{displaymath}
\begin{center}
      \begin{tabular}{cc}
      $ \displaystyle E_a =\Delta H^{\circ\ddagger}+k_B T $
      & $ \displaystyle A = e^1\frac{k_B T}{h} e^{\Delta S^{\circ\ddagger}} $
      \end{tabular}
\end{center}
\item[Bimolecular Reaction] $
        \mathrm{A} + \mathrm{B} \rightleftharpoons [ \mathrm{AB}]^\ddagger
        \rightarrow \text{C}$
      \begin{displaymath}
        k(T)=\nu^\ddagger \bar K^\ddagger=\frac{k_B T}{h} \frac{q_\ddagger(T)/V}{(q_A(T)/V)(q_B(T)/V)}\left
          (\frac{1}{c^\circ}\right )^{-1}
        e^{-\Delta E^\ddagger(0)\beta}
      \end{displaymath}
      \begin{center}
        \begin{tabular}{cc}
        $ \displaystyle E_a  =\Delta H^{\circ\ddagger}+2 k_B T $ & $ \displaystyle
        A  = e^2\frac{k_B T}{h} e^{\Delta S^{\circ\ddagger}} $
      \end{tabular}
      \end{center}
   \end{description}
 \end{center}
 \end{table}
\end{document}